%----------------------------------------------------------------------------------------
%
% COMSC-4173: Mobile Applications - Section 1423
%
% Project Report 4
%
% Created by Thomas W. Crews on Friday, February 3, 2017
%
%
% !!! MUST COMPILE WITH XeLaTeX ENGINE !!!
%
%----------------------------------------------------------------------------------------
%
\documentclass[letterpaper]{article}            % Defines type of document and paper size
%
%----------------------------------------------------------------------------------------
% MODULE IMPORTS
%
\usepackage[english]{babel}                     % Not entirely sure, maybe spellcheck?
\usepackage[utf8]{inputenc}                     % Encoding information (don't touch)
\usepackage[margin=1.0in]{geometry}             % This module honestly scares me
\usepackage{fontspec}                           % Allows advanced font formatting
\usepackage{hyperref}                           % Allows linking between sections
%\usepackage{soul}                              % Basically just for strikeout text
%\usepackage{minted}                            % Syntax highlighting - requires PYGMENTS
%\usepackage{array}                             % Allows advanced table formatting
%\usepackage[table]{xcolor}                     % Allows table shading
%
%----------------------------------------------------------------------------------------
% FONT SETTINGS
%
\setmainfont[Scale=1.1]{Helvetica Neue}         % Define main font and its scale
\setmonofont[Scale=1.1]{Courier New}            % Define monospaced font likewise
%
% Note that Helvetica Neue is only available by default on Apple's Mac OS X 10.6 or newer
% and that Inconsolata is likely not installed by default on any major operating systems.
%
%----------------------------------------------------------------------------------------
% PRINTED DOCUMENT INFORMATION
%
\title{Mobile Applications}                     % Set title
\author{Project Report 4}                       % Set author (subtitle, in this case)
\date{Created by Tommy Crews on \today}         % Date of typesetting
%
%----------------------------------------------------------------------------------------

\begin{document}

\newcommand{\code}[1]{\texttt{#1}}              % I use this for inline code and such

\setcounter{section}{4}							% Use when beginning with subsection

\maketitle                                      % Just prints the title


%\tableofcontents                               % For bigger documents, prints ToC

%----------------------------------------------------------------------------------------
%
% PREDEFINED STRINGS
%
%----------------------------------------------------------------------------------------
%	MAIN
%----------------------------------------------------------------------------------------

% I'll be using these big comment headers to keep track of sections. Subsections are
% delineated by single lines of dashes. Sub-subsections are on their own.

%\section{}

%----------------------------------------------------------------------------------------

\subsection{Please paraphrase or restate your question.}

\subsubsection{Please paraphrase or restate your question from Chapter 1. Now is not the time to be making significant changes to your question. Take advantage of your current time to get something delivered for your project.}

Can a mobile application provide quick and easy network analysis tools to I.T. personnel?

%----------------------------------------------------------------------------------------

\subsection{Start working on your deliverable.}

\subsubsection{What solution are you going to explore?}

My solution will provide a fast and easy-to-use set of networking tools.

\subsubsection{What are the first deliverables?}

The program should have basic interface operations, should not crash under normal use, and should be able to display some amount of helpful information.

\subsubsection{What resources do you need?}

I still need research on using networking modules in Swift, such as \code{WebKit} and \code{NSURL}.

%----------------------------------------------------------------------------------------

\subsection{What did you get done this week for your project?}

\subsubsection{What did you get done?}

This week, I created basic error handling protocol for my application, as well as added features to check for network connections and automatically switch to the Settings application when needed.

\subsubsection{How much time did you spend working on your project?}

This week, I spent around 11 hours working on the project.

%----------------------------------------------------------------------------------------

\subsection{What do you plan to accomplish next week?}

\subsubsection{What is your overall deliverable goal for the next week?}

I plan to resolve any existing issues and push forward adding network features.

\subsubsection{What are the tasks you will complete to reach this goal?}

Swift research will be conducted concerning network capabilities, as well as limitations imposed by iOS. Error handling will be implemented systemwide.

\subsubsection{How much time will you have next week to work on your project?}

10-16 hours.

%----------------------------------------------------------------------------------------

\subsection{What is keeping you from being successful?}

\subsubsection{What has kept you from producing so far in the semester, as listed in Chapter 2?}

My lack of knowledge of the Swift programming language in the field of networking is the biggest obstacle.

\subsubsection{What have you done to fix this?}

I have begun to research official and unofficial documentation sources.

\subsubsection{What is it that is keeping you from being successful next week?}

Nothing at this time is obstructing this project's development.

\subsubsection{Do you need a team mate to work for you and help you, or would you be willing to go help someone else on their project?}

I would be willing to help others with their projects.

\subsubsection{Would you be more productive staying on this project or moving to a different one?}

I believe I am being very productive on this project.

\end{document}

% SDG
