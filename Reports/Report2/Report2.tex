%----------------------------------------------------------------------------------------
%
% COMSC-4173: Mobile Applications - Section 1423
%
% Project Report 2
%
% Created by Thomas W. Crews on Friday, January 20, 2017
%
%
% !!! MUST COMPILE WITH XeLaTeX ENGINE !!!
%
%----------------------------------------------------------------------------------------
%
\documentclass[letterpaper]{article}            % Defines type of document and paper size
%
%----------------------------------------------------------------------------------------
% MODULE IMPORTS
%
\usepackage[english]{babel}                     % Not entirely sure, maybe spellcheck?
\usepackage[utf8]{inputenc}                     % Encoding information (don't touch)
\usepackage[margin=1.0in]{geometry}             % This module honestly scares me
\usepackage{fontspec}                           % Allows advanced font formatting
\usepackage{hyperref}                           % Allows linking between sections
%\usepackage{soul}                              % Basically just for strikeout text
%\usepackage{minted}                            % Syntax highlighting - requires PYGMENTS
%\usepackage{array}                             % Allows advanced table formatting
%\usepackage[table]{xcolor}                     % Allows table shading
%
%----------------------------------------------------------------------------------------
% FONT SETTINGS
%
\setmainfont[Scale=1.1]{Helvetica Neue}         % Define main font and its scale
\setmonofont[Scale=1.1]{Courier New}            % Define monospaced font likewise
%
% Note that Helvetica Neue is only available by default on Apple's Mac OS X 10.6 or newer
% and that Inconsolata is likely not installed by default on any major operating systems.
%
%----------------------------------------------------------------------------------------
% PRINTED DOCUMENT INFORMATION
%
\title{Mobile Applications}                     % Set title
\author{Project Report 2}                       % Set author (subtitle, in this case)
\date{Created by Tommy Crews on \today}         % Date of typesetting
%
%----------------------------------------------------------------------------------------

\begin{document}

\newcommand{\code}[1]{\texttt{#1}}              % I use this for inline code and such

\setcounter{section}{2}							% Use when beginning with subsection

\maketitle                                      % Just prints the title


%\tableofcontents                               % For bigger documents, prints ToC

%----------------------------------------------------------------------------------------
%
% PREDEFINED STRINGS
%
%----------------------------------------------------------------------------------------
%	MAIN
%----------------------------------------------------------------------------------------

% I'll be using these big comment headers to keep track of sections. Subsections are
% delineated by single lines of dashes. Sub-subsections are on their own.

%\section{}

%----------------------------------------------------------------------------------------

\subsection{Please paraphrase or restate your question}

\subsubsection{For 5 points, please paraphrase or restate your question from Chapter 1:}

Can a mobile application provide in-depth network analysis while remaining simple enough for use by average consumers?

%----------------------------------------------------------------------------------------

\subsection{What are some solutions from others?}

\subsubsection{What solution are you interested in, who else has tried these solution, and have others documented their path to a solution?}

Many other similar network analysis tools are available, but are entirely proprietary and not well-documented.

\subsubsection{Do you believe that you will be able to use a path described by someone else?}

Because no documentation or source code exists on these applications' development processes, I likely will not be able to reproduce them.

\subsubsection{Are there limitations to the solutions presented by others?}

Most other solutions are too complicated, too simple, or too unattractive for my question. 

\subsubsection{Will you be able to present an improvement over other solutions?}

My solution will take each minute detail into account. My human interface design is simple, intelligent, and yet extremely informative.

\subsubsection{What are the things about this solution that are interesting to you?}

The realm of networking is one to which I am largely unfamiliar, and thus every step in this project's development is one of discovery and learning.

\subsubsection{What are a few other possible solution you have considered?}

Despite my usual preference (and perhaps against my better judgment), I briefly considered Android development over iOS due to its less restrictive nature.

\subsubsection{What is wrong with the other solutions that makes them less desirable?}

I believe I came to my senses, as a brief amount of research on Android development reminded me of the absolute headache and inefficient mess it promises the developer. 

Additionally, iOS natively contains the majority of application program interfaces required by my project, and the hardware I possess obviouslyleans heavily toward the platform.

\subsubsection{How will producing this solution help you have a better understanding of this class?}

Any time a mobile application is built from the ground-up, its developer is bound to discover new ways of creating and maintaining the project. New methods are discovered and, if properly documented, the efforts of those following them will be greatly simplified.

%----------------------------------------------------------------------------------------

\subsection{What did you get done this week for your project?}

\subsubsection{What did you get done?}

I created marketing materials and mockups for the application, as well as its initial opening screen, whose UI elements are in place. I created formal versioning control of the project, which will be henceforth codenamed \textit{Sycamore}.

\subsubsection{What documentation did you find and what code did you write?}

For this interface, I chose to use radial progress bars. Since these do not exist natively in iOS, I turned to outside resources. As is common with iOS development, a very clean and compact version of one such element was available freely online.

Created by Andrew Bancroft, the element is contained in a single Swift file, and is titled \code{KDCircularProgress.swift}. It can be found here:

\url{https://github.com/kaandedeoglu/KDCircularProgress/blob/master/KDCircularProgress/KDCircularProgress.swift}

Its well-documented implementation instructions can likewise be found here:

\url{https://www.andrewcbancroft.com/2015/07/09/circular-progress-indicator-in-swift/}

Aside from this, the only code I created were test variables and \code{for} loops for the opening screen of the application.

\subsubsection{How much time did you spend working on your project?}

This week, I spent around 12 hours working on the project.

\subsubsection{What new ideas, questions, and concerns have you had?}

I am concerned with my ability to implement automatic network repair functionality due to iOS sandboxing and restriction on arbitrary code execution outside of the application.

\subsubsection{What do you plan to accomplish next week?}

I plan to create the basic interface for the entire application in its current foreseeable plan. Very little code will be written for this iteration.

\subsubsection{What is your overall deliverable goal for the next week?}

I plan to deliver a finished and clean interface, with icon files and no empty placeholders.

\subsubsection{What are the tasks you will complete to reach this goal?}

Interface elements will be created by code if necessary and by image when possible. The interface will aim to match the original mockups as much as possible.

\subsubsection{How much time will you have next week to work on your project?}

10-16 hours.

%----------------------------------------------------------------------------------------

\subsection{What is keeping you from being successful?}

\subsubsection{Please paraphrase what was keeping you from being successful last week, as noted in Chapter 1?}

At this time, determining the limits of my application will be my greatest hindrance.

\subsubsection{Please explain what you did to fix the problems and explain what you asked others to do to fix the problems?}

I realized I wanted to create an application for professionals in my field. I began to ask what my peers would find useful.

\subsubsection{What is it that is keeping you from being successful in your project?}

Nothing at this time is obstructing this project's development.

\subsubsection{Explain how this problem came to be, and please explain if this problem is related to your work in other classes?}

This project actually benefits my work in other classes. It can be referenced to solve problems.

\subsubsection{What are you doing to get this fixed and what have you asked others to do
in order to fix this problem?}

I continue to seek the advice and criticism of my classmates and coworkers throughout the course of this project's development.

\end{document}

% SDG
